\documentclass[11pt]{article}
\usepackage[T1]{fontenc}
\usepackage[utf8]{inputenc}
\usepackage[sort]{natbib}
\usepackage{amsmath}
\usepackage{amssymb}
\usepackage{amsfonts}
\usepackage{bbm}
\usepackage{fancyhdr}
\usepackage{dsfont}
\usepackage{graphicx}
\usepackage{float}
\usepackage{subfig}
\usepackage{listings}
\usepackage{color} %red, green, blue, yellow, cyan, magenta, black, white
\definecolor{mygreen}{RGB}{28,172,0} % color values Red, Green, Blue
\definecolor{mylilas}{RGB}{0,0,255}
\lstset{language=Matlab,%
    %basicstyle=\color{red},
    breaklines=true,%
    morekeywords={matlab2tikz},
    keywordstyle=\color{blue},%
    morekeywords=[2]{1}, keywordstyle=[2]{\color{black}},
    identifierstyle=\color{black},%
    stringstyle=\color{mylilas},
    commentstyle=\color{mygreen},%
    showstringspaces=false,%without this there will be a symbol in the places where there is a space
    numbers=left,%
    numberstyle={\tiny \color{black}},% size of the numbers
    numbersep=9pt, % this defines how far the numbers are from the text
    emph=[1]{for,end,break},emphstyle=[1]\color{blue}, %some words to emphasise
    %emph=[2]{word1,word2}, emphstyle=[2]{style},    
}
\usepackage{enumerate}

\usepackage{framed}
\usepackage{epstopdf}
\usepackage{hyperref}

\usepackage[usenames,dvipsnames,table,tikz]{xcolor} % use colors on table and more
\usepackage{tikz}
\usetikzlibrary{shapes,arrows}

%\usepackage{fancyvrb}
\usepackage[numbered, framed]{mcode}		% codice matlab
\usepackage{enumitem}						% elenco puntato
%\usepackage{pst-sigsys}
\renewcommand{\labelenumi}{\alph{enumi}.} 	% Make numbering in the enumerate environment by letter rather than number (e.g. section 6)
\usepackage[ ]{titlesec}  %
\titleformat{\chapter}[display]
  {\bfseries\Large}
  {\filright\MakeUppercase{\chaptertitlename} \Huge\thechapter}
  {1ex}
  {\titlerule\vspace{1ex}\filleft}
  [\vspace{1ex}\titlerule]

%----- you must not change this -----------------
\oddsidemargin 0cm
\evensidemargin -1.5cm
\topmargin -2.0cm
\textheight 24.0cm
\textwidth 18cm
\parindent=0pt
\parskip 1ex
\renewcommand{\baselinestretch}{1.1}
\renewcommand{\Re}{\operatorname{Re}}
\renewcommand{\Im}{\operatorname{Im}}

\makeatletter
\newenvironment{mcases}[1][l]
 {\let\@ifnextchar\new@ifnextchar
  \left\lbrace
  \def\arraystretch{1.2}%
  \array{@{}l@{\quad}#1@{}}}
 {\endarray\right.}
\makeatother
\pagestyle{fancy}
\setlength{\headheight}{14pt}
%----------------------------------------------------
\title{Wireless Network Security \\ A.A. 2016-2017} 
\author{Roberto Costa} % Author name
\date{\today}

\begin{document}
\begin{center}
	%\horline\\
	\Large A.Y. 2016/17\\
	\huge \textbf{Computer Vision and 3D Graphics Assignment}\\[3mm]
	\begin{framed}
		\Large Student: \textbf{Costa} Roberto\\[2mm]
		\normalsize Student's number: \textbf{1128285}\\
		\Large Teacher: \textbf{Milani} Simone\\[2mm]
	\end{framed}
	%\horline
\end{center}
\section{Assignment}
Plane segmentation using RGBD data. Implement a segmentation strategy from RGBD signals that clusters pixels according to their normals.
\section{Introduction}
The report is structured as follow: \begin{itemize}
\item Algorithm description
\item Results
\item Conclusions
\end{itemize}
\section{Algorithm description}
\begin{itemize}
\item Undistortion procedure
\item Geometric transformation to compute the point cloud X,Y,Z coordinates from the depth map
\item Geometric transformations to project the point cloud the RGB camera image plane
\item Computation of the integral images for the tangential vectors
\item Computation of the local surface normal vectors in Cartesian coordinates
\item Computation of the surface normal vectors in spherical coordinates
\item Quantization of the surface normal vectors
\item Initial segmentation in the normal space
\item Clustering with K-means\begin{itemize}
\item The normals in cartesian coordinates
\item The normals in spherical coordinates
\item The normals and the color information
\item The normals, the color information and the depth
\end{itemize}
\item Clustering evaluation
\end{itemize}
%\begin{figure}[H]
%	\makebox[\textwidth][c]{\includegraphics[width=10cm]{scheme.png}}
%	\caption{a}
%	\label{fig:scheme}
%\end{figure}
\subsection{Parameters}
Window size for normal computation\\
Quantization parameters\\
K-means
\section{Results}
Images
\subsection{Conclusions}
K-means is good and slower, similar results can be obtained with simple quantization of the normals.\\
When adding the RGB information the results improve w.r.t. using only the depth information or the normals.\\
Increasing the window size of the normals allows us to get a good estimate of the floor and the algorithm is real time oriented.\\
Recent improvements have shown that the use of deep learning techniques brings a big improvement.

\end{document}
